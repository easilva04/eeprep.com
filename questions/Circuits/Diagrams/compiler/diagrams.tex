\documentclass{article}
\usepackage{amsmath, amssymb, amsthm}
\usepackage{circuitikz}
\usepackage{tikz}
\usepackage{graphicx}
\usepackage{geometry}
\usepackage{fancyhdr}
\usepackage{hyperref}
\usepackage{titlesec}

\geometry{a4paper, margin=1in}

\begin{document}

% Add your circuit diagrams here
\begin{figure}[h]
    \centering
    \begin{circuitikz}
        % Example circuit diagram
        \draw (0,0) to[battery] (0,2)
              to[resistor] (2,2)
              to[resistor] (2,0)
              -- (0,0);
    \end{circuitikz}
    \caption{Simple circuit with a battery and resistors.}
    \label{fig:simple_circuit}
\end{figure}

\begin{figure}[h]
    \centering
    \begin{circuitikz}
        % Example circuit diagram
        \draw (0,0) to[battery] (0,2)
              to[resistor] (2,2)
              to[resistor] (2,0)
              -- (0,0);
        \draw (2,2) to[short, *-] (3,2)
              to[short] (3,1)
              to[short] (2,1);
    \end{circuitikz}
    \caption{Simple circuit with a battery and resistors.}
    \label{fig:second_simple_circuit}
\end{figure}

\begin{figure}[h]
    \centering
    \begin{circuitikz}
        \draw
        (0,0) to[R, l=$R_1$] (2,0)
        to[R, l=$R_2$] (4,0)
        to[short, -*] (6,0)
        (2,0) to[R, l=$R_3$] (2,-2)
        (4,0) to[R, l=$R_4$] (4,-2)
        (2,-2) to[short, -*] (4,-2)
        (0,0) to[short, -*] (0,2)
        (0,2) to[R, l=$R_5$] (2,2)
        to[R, l=$R_6$] (4,2)
        to[short, -*] (6,2)
        (2,2) to[R, l=$R_7$] (2,4)
        (4,2) to[R, l=$R_8$] (4,4)
        (2,4) to[short, -*] (4,4)
        (0,2) to[short, -*] (0,4)
        (0,4) to[R, l=$R_9$] (2,4)
        to[R, l=$R_{10}$] (4,4)
        to[short, -*] (6,4)
        (6,0) to[short] (6,4);
    \end{circuitikz}
    \caption{Complex resistor network.}
    \label{fig:resistor_network}
\end{figure}

\begin{figure}[h]
    \centering
    \begin{circuitikz}
        \draw
        (0,0) to[L, l=$L_1$] (2,0)
        to[C, l=$C_1$] (4,0)
        to[short, -*] (6,0)
        (2,0) to[L, l=$L_2$] (2,-2)
        (4,0) to[C, l=$C_2$] (4,-2)
        (2,-2) to[short, -*] (4,-2)
        (0,0) to[short, -*] (0,2)
        (0,2) to[L, l=$L_3$] (2,2)
        to[C, l=$C_3$] (4,2)
        to[short, -*] (6,2)
        (2,2) to[L, l=$L_4$] (2,4)
        (4,2) to[C, l=$C_4$] (4,4)
        (2,4) to[short, -*] (4,4)
        (0,2) to[short, -*] (0,4)
        (0,4) to[L, l=$L_5$] (2,4)
        to[C, l=$C_5$] (4,4)
        to[short, -*] (6,4)
        (6,0) to[short] (6,4);
    \end{circuitikz}
    \caption{Inductor and capacitor network.}
    \label{fig:inductor_capacitor_network}
\end{figure}

\begin{figure}[h]
    \centering
    \begin{circuitikz}
        \draw
        (0,0) to[V, l=$V_1$] (0,4)
        (0,4) to[R, l=$R_1$] (4,4)
        (4,4) to[L, l=$L_1$] (4,0)
        (4,0) to[C, l=$C_1$] (0,0)
        (2,4) to[R, l=$R_2$] (2,2)
        (2,2) to[L, l=$L_2$] (4,2)
        (4,2) to[C, l=$C_2$] (2,0)
        (2,0) to[R, l=$R_3$] (0,2)
        (0,2) to[L, l=$L_3$] (2,2);
    \end{circuitikz}
    \caption{Complex circuit with voltage source, resistors, inductors, and capacitors.}
    \label{fig:complex_circuit}
\end{figure}

\end{document}
